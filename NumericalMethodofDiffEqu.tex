\documentclass[12pt, a4paper]{article}
\usepackage{ctex}


\usepackage{graphicx}%图片包
\graphicspath{{figures/}}
%\includegraphics[scale=0.6]{.eps}\\

\usepackage{amsmath}%数学公式包
\usepackage{amssymb}%特殊符号包
\usepackage{bm}%加粗数学符号 \bm{math expression}

\usepackage{color}%Excel2LaTeX辅助包
\usepackage{booktabs}
\usepackage{multirow}
%\resizebox{120mm}{100mm}{} 调整表格大小
\usepackage{array}

\usepackage{listings}%插入代码
\lstset{language=Matlab}
\lstset{breaklines}%自动将长的代码行换行排版
\lstset{extendedchars=false}%解决代码跨页时,章节标题,页眉等汉字不显示的问题

\usepackage{ulem}

\title{数理方程习题}
\author{中国语言文学系常代表办公室}
\renewcommand{\today}{\number\year /\number\month /\number\day}

\usepackage{subfigure}
\graphicspath{{pic/}}
\usepackage{geometry}
\geometry{left=2cm, right=2cm, top=2.5cm, bottom=2.5cm}

\begin{document}
	\maketitle
	\iffalse
	%多行公式 &对齐
	\begin{align}
	\\
	&=
	
	\end{align}
	
	%圆括号矩阵
	$$A_1=
	\begin{pmatrix}
	&   &   &   &  \\
	&   &   &   &  \\
	%\dots \ddots \vdots
	\end{pmatrix}
	$$
	
	%靠左
	\begin{flushleft}
		\\
	\end{flushleft}
	
	%大括号表示
	$$x_{k+1} =
	\begin{cases}
	x_k+s_k & \text{if } \rho_k>\eta_1\\
	x_k & \text{otherwise} 
	\end{cases}$$
	
	%文字样式
	\uline{下划线}
	\uuline{双下划线}
	\uwave{波浪线}
	\sout{中间删除线}
	\xout{斜删除线}
	\dashuline{虚线}
	\dotuline{加点}
	\fi
	\newpage
	\tableofcontents
	\newpage
	\section{调和方程}
	\subsection{$\S_{1.3}$第2题}
	\kaishu{}
	对一般的椭圆型方程$$
	\sum_{i,j=1}^{n}a_{ij}\frac{\partial^2 u}{\partial x_i \partial x_j}+\sum_{i=1}^{n}b_i \frac{\partial u}{\partial x_i}+cu=0$$
	其中$(a_{ij})$为正定阵,$c\le 0$.证明弱极值原理,强极值原理成立.\\
	
	\songti{}\uuline{解答}:
	
	首先我们必须叙述清楚一般椭圆型方程极值原理的含义。\\
	
	\uwave{弱极值原理}:
	
	若$u\in C^2(B_1) \cap C(\bar{B_1})$,满足一般椭圆方程,则$u$不能在$B_1$内部取到非负最大值/非正最小值.

	\uwave{强极值原理}:
	
	若$u\in C^2(B_1) \cap C(\bar{B_1})$,满足一般椭圆方程.设$x^0\in \partial B_1$,使得$$
	u(x^0)>u(x),\forall x \in B_1$$
	则有$$
	\liminf_{t \to 0^+} \frac{u(x^0)-u(x^0-t\bm{n})}{t} >0 $$
	其中$\bm{n}$为$\partial B_1$在$x_0$处的单位外法向量.\\
	
	证明:
	
	先证弱极值原理.记题中方程左端为$L(u)$,其中$L$是椭圆型微分算子,容易验证它是线性的.则椭圆型方程可以简写成$L(u)=0$.只证明非负最大值不能在内部取得.我们先考虑$L(u)>0$时的情形,然后用摄动的方法推证$L(u)=0$时也成立.
	
	$L(u)>0$,用反证法,假设$\exists x_0 \in B_1$,使得$u$在$x_0$处达到了非负最大值.那么有
	$$
	\begin{cases}
	\left.u\right|_{x_0} &\ge 0  \\
	\left. \frac{\partial u}{\partial x_i}\right|_{x_0} &=0 , \forall i  \\ \left.H=\left( \frac{\partial^2 u}{\partial x_i \partial x_j}\right)_{i,j} \right|_{x_0}&\le 0  
	\end{cases}$$
	但是,在${x_0}$处,有$$
	\sum_{i,j=1}^{n}a_{ij}\frac{\partial^2 u}{\partial x_i \partial x_j}>-cu\ge 0
	$$
	记$A=(a_{ij})$,则它是正定阵,那么在${x_0}$处,有
	\begin{align}
		&\qquad \sum_{i,j=1}^{n}a_{ij}\frac{\partial^2 u}{\partial x_i \partial x_j}> 0\notag\\
		&\Rightarrow tr(AH)>0\notag \\
		&\Rightarrow tr(C^T CH)>0 ,\exists C \in \mathbb{R}^{n \times n}\notag \\
		&\Rightarrow tr(CHC^T)>0 \notag\\
		&\Rightarrow \text{H正定,矛盾!}\notag 
	\end{align}
	
	下面考虑一般的情形,取$h=e^{kx_1}$,其中$k>0$,考察函数$u_{\epsilon}=u+\epsilon h$,有$L(u_{\epsilon})=L(u+\epsilon h)=L(u)+\epsilon L(h)=0+\epsilon(a_{11}k^2+b_1 kh+ch)h$,我们希望有$L(u+\epsilon h)>0$,可以取$h$足够大,那么对足够小的$\epsilon$,$u_{\epsilon}$的非负最大值恒在边界上取得,考虑到$u_{\epsilon}$的连续性,令$\epsilon \to 0$,即得证.
	
	下面证明强极值原理.我们希望构造函数$w=u-v$,对$w$使用弱极值原理.而函数$v$满足
	$$\begin{cases}
		\left.v\right|_{\partial B_1} =0\\
		L(v)<0,in \quad B_1\backslash B_{\frac{1}{2}} \\ 
		\frac{\partial v}{\partial \bm{n}}>0
	\end{cases}$$
	可以取$$
	v=v_\alpha=e^{-\alpha}-e^{-\alpha r^2}$$
	取$\alpha$足够大就可以满足上面的要求,剩下的工作和书上的做法是类似的.\\
	
	\subsection{$\S_{1.2}$第9题}
	\kaishu{}
	设$u$为带状区域$\Sigma=\{(x_1,x_2)||x_1|\le 1, \quad -\infty<x_2< \infty\}$上的非负调和函数,证明:$$
	u(x)\le u(0)e^{C|x_2|},\quad \forall |x_1|\le \frac{1}{4},\quad -\infty<x_2< \infty		$$
	其中常数$C$不依赖于$u$.(提示:Harnack不等式)\\
	
	\songti{}
	\uuline{解答}:
	
	对于任意的$x=(x_1,x_2)\in \Sigma$,取半径为$\frac{3}{16}$的圆,用“滚圆法”折线连接$(x_1,x_2),(x_1,0),(0,0)$三点,假设总共需要N+2个圆$\{B_k\}_{k=1}^{N+2}$,那么第1,N,N+2个圆的圆心即为这三个点,记这些圆的圆心分别为$\{x^k\}_{k=1}^{N+2}$.
	
	
	此时满足注1.2.3的条件,可以使用Harnack不等式的推论,有
	\begin{align}
		u(x)&=u(x^1)\notag \\
		&\le C(n)u(x^2) \notag\\
		&\le (C(n))^2 u(x^3) \notag \\
		&\le \dots \notag \\
		&\le (C(n))^{N+1}u(x^{N+2}) \notag \\
		&= (C(n))^{N+1}u(0) \notag
	\end{align}\\
	且显然有
	\begin{align}
		&\qquad (N-1)\frac{3}{16} \le |x_2|	\notag \\
		&\Rightarrow N \le \frac{16}{3}|x_2| +1 \notag \\
		&\Rightarrow N+1 \le \frac{16}{3}|x_2| +2 \notag \\
		&\Rightarrow u(x) \le (C(n))^{\frac{16}{3}|x_2| +2}u(0) \notag 
	\end{align}\\
	原题应当有误,右端还应该多一个常数.\\
	
	\subsection{$\S_{1.2}$第10题}
	\kaishu{}
	设$u \in C^2(\overline{\mathbb{R}_{+}^{n}})$为$\overline{\mathbb{R}_{+}^n}=\{(x_1,\dots,x_n)\in \mathbb{R}^n|x_n \ge 0\}$上的调和函数.若满足$u(x_1,\dots,x_{n-1},0)=0$且$u(x)$为有界函数,证明$u\equiv 0$.(提示:延拓为全空间调和函数)若把$u$改为下有界函数,试举出例子使得结论不成立.\\
	
	\songti{}
	\uuline{解答}:
	
	对$x=(x_1,x_2,\dots,x_n)\in \mathbb{R}^n$,定义$\bar{x}=(x_1,x_2,\dots,-x_n)$.	

	我们给出$u$的全空间调和延拓的构造,然后应用Liouville型定理即可.$$
	v(x)=\begin{cases}
	u(x) & \text{if } x_n \ge 0\\
	\overline{u(\bar{x})} & \text{if } x_n < 0
	\end{cases}$$
	若把$u$改为下有界函数,结论不成立,反例有$u(x)=x_n$.
	
	\subsection{$\S_{1.2}$第11题}
	\kaishu{}
	若$u$为$\mathbb{R}^2$上的调和函数.设$0<a\le b\le c$满足$b^2=ac$,证明$$
	\int_{\mathbb{S}^1} u(x^0+a\omega)u(x^0+c\omega)dS_{\omega}=\int_{\mathbb{S}^1}
	u^2 (x^0+b\omega)dS_{\omega}	$$
	(提示:固定$b$后,对上式左端求导.)上述结论对一般维数$\mathbb{R}^n$也成立,有兴趣的同学可以自己证明.\\
	
	\songti{}
	\uuline{解答}:
	
	固定b,记$f(a)=\int_{\mathbb{S}^1} u(x^0+a\omega)u(x^0+\frac{b^2}{a}\omega)dS_{\omega}$,只要证明$f^{'}(a)=0$即可.
	\begin{align}
		&\qquad f^{'}(a)=\int_{\mathbb{S}^1}\omega\cdot \nabla u(x^0+a\omega)u(x^0+\frac{b^2}{a}\omega)dS_{\omega} -\frac{b^2}{a^2}\int_{\mathbb{S}^1}\omega\cdot \nabla u(x^0+\frac{b^2}{a}\omega) u(x^0+a\omega)dS_{\omega}\notag \\
		&=\frac{1}{a} \left(\int_{\mathbb{S}^1}a\omega\cdot \nabla u(x^0+a\omega)u(x^0+\frac{b^2}{a}\omega)dS_{\omega} -\int_{\mathbb{S}^1}\frac{b^2}{a}\omega\cdot \nabla u(x^0+\frac{b^2}{a}\omega) u(x^0+a\omega)dS_{\omega}\right)		\notag\\
		&=\frac{1}{a} \left(\int_{\mathbb{S}^1}\frac{\partial u(x^0+aw)}{\partial n_{\omega}}u(x^0+\frac{b^2}{a}\omega)dS_{\omega} -\int_{\mathbb{S}^1}\frac{\partial u(x^0+\frac{b^2}{a}w)}{\partial n_{\omega}} u(x^0+a\omega)dS_{\omega}\right)		\notag\\
		&\stackrel{\text{Green第二公式}}{=}\frac{1}{a} \left(\iint_{B_1}\Delta u(x^0+aw)u(x^0+\frac{b^2}{a}\omega)d\omega -\iint_{B_1} \Delta u(x^0+\frac{b^2}{a}w) u(x^0+a\omega)d\omega\right)		\notag\\
		&=0 \notag
	\end{align}
	
	\subsection{$\S_{1.3}$第4题}
	
	
\end{document}